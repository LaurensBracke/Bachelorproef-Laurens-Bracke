%---------- Inleiding ---------------------------------------------------------

\section{Introductie} % The \section*{} command stops section numbering
\label{sec:introductie}

Studenten hebben dezer dagen zoveel keuze bij het aankopen van hardware. Ze kunnen een smartphone kopen, die een tablet kan worden, of een tablet die de mogelijkheden bevat van een laptop. Dan spreken we nog niet over het besturingssysteem: werkt het op Android, Windows, Apple of Linux? U ziet dat studenten de weg kunnen kwijtraken en gemakkelijk meegaan met de nieuwste hypes en trends. Elke student heeft door deze immense keuze een unieke verzameling. En iedere student gaat op een andere manier om met zijn apparaten.

De doelstelling van dit onderzoek is om na te gaan wat studenten doen met hun aankopen: laten ze deze thuis of nemen ze deze juist mee? En als ze deze apparaten meenemen naar de les, wat is de functie dan van deze apparaten? Gebruiken studenten ze als een manier om vragen in de les te vermijden, om vanuit de les toch in contact te blijven met sociale media of enkel als back-up in een noodgeval?

Hoe gaan scholen om met al deze nieuwe snufjes? Het is natuurlijk de bedoeling dat studenten die naar de les komen opletten, maar als zelfs al een horloge contact kan maken met het internet en sociale media, moeten scholen deze dan ook beginnen verbieden? Door eerst een kleine enquête te houden bij studenten, kunnen stukken van de hoofdvraag en deelvragen beantwoord worden. Door daarna werkelijk klassen te observeren en het gedrag van studenten, kunnen we zeker zijn van wat in de enquêtes naar boven kwam en kan men scholen en leerkrachten helpen bij het opwaarderen van de lessen. en onderzoeksvraag/-vragen.
%---------- Stand van zaken ---------------------------------------------------

\section{State-of-the-art}
\label{sec:state-of-the-art}

Het onderzoek van \textcite{3} voorspelde dat rond het jaar 2015 normaal iedereen in een klas in het bezit zou zijn van een mobiel apparaat. Het onderzoek wijst ook naar de voordelen voor lesgevers wanneer hun studenten het \emph{juiste} apparaat voor zich hebben liggen. Er wordt ook meegegeven in het onderzoek dat zes jaar geleden de personen vooral hun device gebruikten voor communicatie en games. Men had toen nog niet het volledige potentieel door hiervan. 

\textcite{2} en \textcite{1} tonen verschillende trends aan. Ten eerste is duidelijk dat de verkoop van smartphones nog steeds stijgt, terwijl de verkoop van tablets daalt. Ten tweede is duidelijk dat het verschepen van laptops, desktops en tablets wereldwijd sterk verminderd is t.o.v. 2013. Vooral de desktop-markt stort in. Mensen willen on-the-move op het internet actief zijn.

\textcite{4} en \textcite{5} zitten bij dit voorstel om aan te tonen dat het volgens de grote voorspellers al heel lang duidelijk was dat mensen steeds meer en meer kiezen voor een mobile device. 

Mijn bronnen tonen één gemeenschappelijk punt aan: we kunnen niet meer zonder onze hardware in het dagelijks leven. Maar het toont ook aan dat leerkrachten het steeds moeilijker zullen krijgen om hun studenten bij de les te houden. In dit onderzoek willen we zoeken naar oorzaken en oplossingen hiervoor..

%---------- Methodologie ------------------------------------------------------
\section{Methodologie}
\label{sec:methodologie}

Door het verspreiden van een enquête onder de groep studenten aan de Hogeschool Gent en de Haagse Hogeschool in Nederland, proberen we zoveel mogelijk data te verzamelen van elke student: hardware-configuratie enerzijds en het gebruik in de les daarvan en eigen reflectie hierover anderzijds (schrijfmateriaal, sociale media, oplossingen zoeken voor vragen...). We moeten ervoor zorgen dat we evenveel jongens als meisjes benaderen, en ook dat we zowel technische als sociale richtingen ondervragen. 

Nadat alle data is binnengekomen, doen we onderzoek in de klaslokalen zelf om de resultaten van de enquêtes kracht bij te zetten. We laten aan elke hogeschool 3-4 leerkrachten van verschillende richtingen een theoretische les geven aan verschillende klasgroepen. Bij de ene zeggen we niets en laten we alles toe (laptops, gsm's...), bij de andere klasgroep geven we dezelfde les en geven we duidelijk aan dat alle smartphones in het zicht liggen van de leerkracht op tafel ofwel afgegeven worden voor de les. Op het einde van de les kunnen we dan de graad van opletten toetsen door onverwacht elke student 10 korte vragen te laten invullen over de les die ze net hebben gekregen.  

Als beide zijn afgerond, kunnen we de twee onderzoeken naast elkaar leggen en de theorie-trends uit de enquêtes aan de praktijk toetsen. Een goeie vragenlijst, een statisch programma zoals Rstudio of Excel en medewerking van enkele leerkrachten zullen voldoende zijn om dit onderzoek te voeren.

%---------- Verwachte resultaten ----------------------------------------------
\section{Verwachte resultaten}
\label{sec:verwachte_resultaten}

Op basis van de state-of-the-art en het voorgevoel verwachten we dat de eigen devices van de studenten ervoor zorgen dat de student minder aanwezig is bij de les, rap afgeleid en minder begrijpt van de les. Studenten die minder bezig zijn met hun apparaten tijdens de les, zullen ook betere cijfers halen. De verwachting is ook dat mannelijke studenten rapper zijn afgeleid, en dat ook durven toegeven in de enquête. Daarnaast zullen studenten er mee kunnen leven dat ofwel hun activiteit op het schoolnetwerk gevolgd wordt, ofwel enkel apparaten van de school gebruikt mogen worden tijdens de lessen.

%---------- Verwachte conclusies ----------------------------------------------
\section{Verwachte conclusies}
\label{sec:verwachte_conclusies}

We hopen te kunnen concluderen dat scholen nog veel meer kunnen doen om de aandacht van de studenten hoog te houden. Strenge regels of niet, leerlingen zullen zich toch altijd aanpassen aan wat de school hen oplegt. Leerkrachten moeten geholpen worden in hun doel om studenten die in de les zitten, allemaal mee te krijgen in hun verhaal. Dit kan door ofwel strenger optreden tegen het BYOD-principe, ofwel het BYOD-principe in de armen te sluiten en deze apparaten meer te integreren in de lessen.   

Het onderzoek moet nuttige informatie opleveren voor de scholen en bedrijven. Scholen kunnen door deze info inspringen op de vragen en noden van de jongeren. Scholen kunnen nadenken over wat noodzakelijk zal zijn in de komende jaren qua investeringen in infrastructuur of welke elektronica de studenten nodig zullen hebben voor hun studies kwalitatief af te ronden. In de toekomst kunnen lessen aangepast worden en veel interactiever zijn wanneer de devices van studenten ten volle worden gebruikt. 

