%==============================================================================
% Sjabloon onderzoeksvoorstel bachelorproef
%==============================================================================
% Gebaseerd op LaTeX-sjabloon ‘Stylish Article’ (zie voorstel.cls)
% Auteur: Jens Buysse, Bert Van Vreckem

% TODO: Compileren document:
% 1) Vervang ‘naam_voornaam’ in de bestandsnaam door je eigen naam, bv.
%    buysse_jens_voorstel.tex
% 2) latexmk -pdf naam_voornaam_voorstel.tex
% 3) biber naam_voornaam_voorstel
% 4) latexmk -pdf naam_voornaam_voorstel.tex (1 keer)

\documentclass[fleqn,10pt]{voorstel}
\usepackage[english,dutch]{babel}

\usepackage[backend=biber,style=apa]{biblatex}
\DeclareLanguageMapping{dutch}{dutch-apa}
\addbibresource{biblio.bib}

\usepackage{lipsum}

\usepackage{hyperref} % Required for hyperlinks
\hypersetup{hidelinks,colorlinks,breaklinks=true,urlcolor=color1,citecolor=color2,linkcolor=color2,bookmarksopen=false,pdftitle={Title},pdfauthor={Author}}
%------------------------------------------------------------------------------
% Metadata over het artikel
%------------------------------------------------------------------------------

\JournalInfo{HoGent Bedrijf en Organisatie} % Journal information
\Archive{Bachelorproef 2017-2018} % Additional notes (e.g. copyright, DOI, review/research article)

%---------- Titel & auteur ----------------------------------------------------

% TODO: geef werktitel van je eigen voorstel op
\PaperTitle{Effect van devices (BYOD) op leerkracht en student}
\PaperType{Onderzoeksvoorstel Bachelorproef} % Type document

% TODO: vul je eigen naam in als auteur, geef ook je emailadres mee!
\Authors{Laurens Bracke (3B1) \textsuperscript{1}} % Authors
\affiliation{\textbf{Contact:}
  \textsuperscript{1} \href{mailto:laurens.bracke.w1090@student.hogent.be}{laurens.bracke.w1090@student.hogent.be}}

%---------- Abstract ----------------------------------------------------------

  \Abstract{  	
	Studenten aan de hogeschool hebben tegenwoordig een volledig arsenaal aan elektronische apparaten bij zich als ze naar school komen: een smartphone in de jas, een tablet en een laptop in de rugzak en vaak nog een desktop thuis. Soms is dit een verplichting van de studierichting, soms is het noodzakelijk voor taken en soms is het gewoon een 'extra' cadeautje voor zichzelf. Maar wat heeft nu welke student bij zich? En heeft het meepakken van al deze apparaten naar de les wel zijn nut? Het doel van deze opdracht is om via enquêtes  (gericht aan studenten in het hoger onderwijs) en een testopstelling te weten te komen wat zij dagelijks meebrengen naar de les, wat ze werkelijk gebruiken tijdens de les en of deze hun helpen om betere punten te behalen en hun aandacht bij de les te houden. Het hoofddoel is om met al deze gegevens te onderzoeken of het concept BYOD (Bring Your Own Device) de studenten (en de scholen) wel degelijk vooruithelpt, en wat de studenten zelf vinden van de scholen hun politiek over smartphones en laptops in de les. Mogelijke deelvragen: Hoe gaan Nederlandse scholen om met apparaten van hun studenten, kunnen vrouwen beter hun focus houden dan mannen met al deze verleidelijke elektronica en sociale media rond hen, kunnen studenten uit technische richtingen meer hun aandacht bij de les houden, moeten scholen meer inspringen op de apparatuur van hun leerlingen en interactievere lessen aanbieden... De verwachting is dat deze apparaten hun nut hebben bij praktijklessen die strikt gestuurd worden door de leerkracht, maar dat bij theorielessen beter het bovenhalen van smartphones of laptops verboden beperkt wordt. Scholen kunnen verder uit dit onderzoek halen wat de noden zijn voor de studenten, wat noden zijn voor zichzelf en of het niet beter is voor de scores van de studenten om betere regels te maken voor de hele school voor het gebruik van eigen devices. Ook kan dit resultaat dienen voor bedrijven, die het BYOD-principe willen toepassen op hun werknemers en deze strikter willen opvolgen.
  }

%---------- Onderzoeksdomein en sleutelwoorden --------------------------------
% TODO: Sleutelwoorden:
%
% Het eerste sleutelwoord beschrijft het onderzoeksdomein. Je kan kiezen uit
% deze lijst:
%
% - Mobiele applicatieontwikkeling
% - Webapplicatieontwikkeling
% - Applicatieontwikkeling (andere)
% - Systeem- en netwerkbeheer
% - Mainframe
% - E-business
% - Databanken en big data
% - Machine learning en kunstmatige intelligentie
% - Andere (specifieer)
%
% De andere sleutelwoorden zijn vrij te kiezen

\Keywords{Systeembeheer. Hardware --- Studenten --- Devices} % Keywords
\newcommand{\keywordname}{Sleutelwoorden} % Defines the keywords heading name

%---------- Titel, inhoud -----------------------------------------------------
\begin{document}

\flushbottom % Makes all text pages the same height
\maketitle % Print the title and abstract box
\tableofcontents % Print the contents section
\thispagestyle{empty} % Removes page numbering from the first page

%------------------------------------------------------------------------------
% Hoofdtekst
%------------------------------------------------------------------------------

%---------- Inleiding ---------------------------------------------------------

\section{Introductie} % The \section*{} command stops section numbering
\label{sec:introductie}

Studenten hebben dezer dagen zoveel keuze bij het aankopen van hardware. Ze kunnen een smartphone kopen, die een tablet kan worden, of een tablet die de mogelijkheden bevat van een laptop. Dan spreken we nog niet over het besturingssysteem: werkt het op Android, Windows, Apple of Linux? U ziet dat studenten de weg kunnen kwijtraken en gemakkelijk meespringen op de nieuwste trends. Elke student heeft door deze immense keuze een unieke verzameling. En iedere student gaat op een andere manier om met zijn apparaten.

De doelstelling van dit onderzoek is om na te gaan wat studenten doen met hun aankopen: laten ze deze thuis of nemen ze deze juist mee? En als ze deze apparaten meenemen naar de les, wat is de functie dan van deze apparaten? Gebruiken studenten ze als een manier om vragen in de les te vermijden, om vanuit de les toch in contact te blijven met sociale media of enkel als back-up in een noodgeval?

Hoe gaan scholen om met al deze nieuwe snufjes? Het is natuurlijk de bedoeling dat studenten die naar de les komen opletten, maar als zelfs al een horloge contact kan maken met het internet en sociale media, moeten scholen deze dan ook beginnen verbieden? Door eerst een kleine enquête te houden bij studenten, kunnen stukken van de hoofdvraag en de deelvragen beantwoord worden. Door daarna werkelijk klassen te observeren en het gedrag van studenten, kunnen we zeker zijn van wat in de enquêtes naar boven kwam en kan men scholen en leerkrachten helpen bij het opwaarderen van de lessen.

%---------- Stand van zaken ---------------------------------------------------

\section{State-of-the-art}
\label{sec:state-of-the-art}

Het onderzoek van \textcite{3} voorspelde dat rond het jaar 2015 normaal iedereen in een klas in het bezit zou zijn van een mobiel apparaat. Het onderzoek wijst ook naar de voordelen voor lesgevers wanneer hun studenten het \emph{juiste} apparaat voor zich hebben liggen. Er wordt ook meegegeven in het onderzoek dat zes jaar geleden de personen vooral hun device gebruikten voor communicatie en games. Men had toen nog niet het volledige potentieel door hiervan. 

\textcite{2} en \textcite{1} tonen verschillende trends aan. Ten eerste is duidelijk dat de verkoop van smartphones nog steeds stijgt, terwijl de verkoop van tablets daalt. Ten tweede is duidelijk dat het verschepen van laptops, desktops en tablets wereldwijd sterk verminderd is t.o.v. 2013. Vooral de desktop-markt stort in. Mensen willen on-the-move op het internet actief zijn.

\textcite{4} en \textcite{5} zitten bij dit voorstel om aan te tonen dat het volgens de grote voorspellers al heel lang duidelijk was dat mensen steeds meer en meer kiezen voor een mobile device. 

Mijn bronnen tonen één gemeenschappelijk punt aan: we kunnen niet meer zonder onze hardware in het dagelijks leven. Maar het toont ook aan dat leerkrachten het steeds moeilijker zullen krijgen om hun studenten bij de les te houden. In dit onderzoek willen we zoeken naar oorzaken en oplossingen hiervoor.

%---------- Methodologie ------------------------------------------------------
\section{Methodologie}
\label{sec:methodologie}

Door het verspreiden van een enquête onder de groep studenten aan de Hogeschool Gent en de Haagse Hogeschool in Nederland, proberen we zoveel mogelijk data te verzamelen van elke student: hardware-configuratie enerzijds en het gebruik in de les daarvan en eigen reflectie hierover anderzijds (noteermateriaal, sociale media, oplossingen zoeken voor vragen...). We moeten ervoor zorgen dat we evenveel jongens als meisjes benaderen, en ook dat we zowel technische als sociale richtingen ondervragen. 

Nadat alle data is binnengekomen, doen we onderzoek in de klaslokalen zelf om de resultaten van de enquêtes kracht bij te zetten. We laten aan elke hogeschool 3-4 leerkrachten van verschillende richtingen een theoretische les geven aan verschillende klasgroepen. Bij de ene zeggen we niets en laten we alles toe (laptops, gsm's...), bij de andere klasgroep geven we dezelfde les en geven we duidelijk aan dat alle smartphones duidelijk in beeld liggen van de leerkracht op tafel ofwel afgegeven worden voor de les. Op het einde van de les kunnen we dan de graad van opletten toetsen door onverwacht elke student 10 korte vragen te laten invullen over de les die ze net hebben gekregen.  

Als beide zijn afgerond, kunnen we de twee onderzoeken naast elkaar leggen en de theorie-trends uit de enquêtes aan de praktijk toetsen. Een goeie vragenlijst, een statisch programma zoals Rstudio en medewerking van enkele leerkrachten zullen voldoende zijn om dit onderzoek te voeren.

%---------- Verwachte resultaten ----------------------------------------------
\section{Verwachte resultaten}
\label{sec:verwachte_resultaten}

Op basis van de state-of-the-art en het voorgevoel verwachten we dat de eigen devices van de studenten ervoor zorgen dat de student minder aanwezig is bij de les, rap afgeleid en minder begrijpt van de les. Studenten die minder bezig zijn met hun apparaten tijdens de les, zullen ook betere cijfers halen. De verwachting is ook dat mannelijke studenten rapper zijn afgeleid, en dat ook durven toegeven in de enquête. Daarnaast zullen studenten er mee kunnen leven dat ofwel hun activiteit op het schoolnetwerk getrackt wordt, ofwel enkel apparaten van de school gebruikt mogen worden tijdens de lessen.

%---------- Verwachte conclusies ----------------------------------------------
\section{Verwachte conclusies}
\label{sec:verwachte_conclusies}

We hopen te kunnen concluderen dat scholen nog veel meer kunnen doen om de aandacht van de studenten hoog te houden. Strenge regels of niet, leerlingen zullen zich toch altijd aanpassen aan wat de school hen oplegt. Leerkrachten moeten geholpen worden in hun doel om studenten die in de les zitten, allemaal mee te krijgen in hun verhaal. Dit kan door ofwel strenger optreden tegen het BYOD-principe, ofwel het BYOD-principe in de armen te sluiten en deze apparaten meer te integreren in de lessen.   

Het onderzoek moet nuttige informatie opleveren voor de scholen en bedrijven. Scholen kunnen door deze info inspringen op de vragen en noden van de jongeren. Scholen kunnen nadenken over wat noodzakelijk zal zijn in de komende jaren qua investeringen in infrastructuur of welke elektronica de studenten nodig zullen hebben voor hun studies kwalitatief af te ronden. In de toekomst kunnen lessen aangepast worden en veel interactiever zijn wanneer de devices van studenten ten volle worden gebruikt. 

%------------------------------------------------------------------------------
% Referentielijst
%------------------------------------------------------------------------------
% TODO: de gerefereerde werken moeten in BibTeX-bestand ``biblio.bib''
% voorkomen. Gebruik JabRef om je bibliografie bij te houden en vergeet niet
% om compatibiliteit met Biber/BibLaTeX aan te zetten (File > Switch to
% BibLaTeX mode)

\phantomsection
\printbibliography

\end{document}