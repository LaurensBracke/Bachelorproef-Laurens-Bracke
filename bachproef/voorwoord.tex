%%=============================================================================
%% Voorwoord
%%=============================================================================

\chapter*{Woord vooraf}
\label{ch:voorwoord}

Ik heb ervoor gekozen om onderwerpen te selecteren die alomtegenwoordig zijn in onze maatschappij: de smartphone, sociale media, het internet... Het beeld van de hogeschoolstudent is in slechts 20 jaar compleet veranderd: praktisch elke student heeft nu een smartphone, deelt al zijn toffe ideeën of vragen over een examen op sociale media en studeert zijn lessen door naar een YouTube-filmpje te kijken van iemand die aan de andere kant van de wereld woont. Voor mij en mogelijks voor andere studenten lijkt het daardoor dat iedereen kan bijleren op verschillende manieren. Studenten die het moeilijk hebben om zich te concentreren in een volle aula, kunnen zich mogelijks thuis achter hun computerscherm beter focussen op de materie. Er zijn heel veel positieve kanten aan de technologische vooruitgang.

Maar als student zie ik vaak ook de negatieve kant van deze technologie. Studenten nemen hun laptop of smartphone mee naar de les, om vervolgens de hele tijd te gamen of te chatten met hun beste vriend of vriendin. Tijdens middagpauzes is het interessanter om sociale media te checken dan een normaal gesprek te voeren met je medestudenten. Studenten die afgeleid zijn door een nieuwe melding op hun scherm en zo hun aandacht van de les verliezen, terwijl er net iets verteld werd over het examen. Waar je vroeger enkel pen en papier voor je neus had liggen, heb je nu een volledig arsenaal aan snufjes en elektronica op je schoolbank liggen om even te kunnen ontsnappen uit die saaie les.

De volwassen student is natuurlijk oud en wijs genoeg om hier zelf de controle over te bewaren. Toch heb ik het gevoel dat studenten die alle technologie in hun broekzakken en rugzak laten tijdens de les, op het einde van de rit ook met een comfortabeler gevoel de examenperiode ingaan en betere resultaten halen. Ook lijkt in mijn ogen dat studenten die veel uren spenderen achter een computer- of smartphonescherm hun gemiddeld punt lager is dan een voorbeeldige student, die enkel gefocust is op de leerkracht en de materie die onderwezen wordt. Dit gevoel wou ik testen voor mijn scriptie, en kijken of er enige vorm van waarheid inzit.

Door middel van een enquête wou ik studenten van zowel de Hogeschool Gent als de Haagse Hogeschool ondervragen over hun gewoontes en schoolresultaten: hoeveel keer ze afgeleid zijn in een les, hoeveel uur ze spenderen per dag achter een klein schermpje, wat hun gemiddeld resultaat is, of ze veel herexamens hebben elk jaar... Daarnaast wou ik ook in de praktijk testen of studenten wel degelijk hun mobieltjes of laptops nodig hebben in de les. Ik denk dat wanneer de verleiding wegvalt tijdens een les, de kennis die wordt opgenomen door de volledige klas groter is.

Voor al dit te verwezenlijken heb ik mijn uiterste best gedaan om zoveel mogelijk informatie te verzamelen, met studenten te gaan praten en leerkrachten te overtuigen van het nut van dit onderzoek. Daarom zou ik graag enkele mensen bedanken die me hebben geholpen om dit onderzoek tot een goed einde te brengen.

Als eerste dank ik mijn promotor Noémie Slaats. Zij heeft me van kortbij opgevolgd en heeft me altijd een duw in de juiste richting gegeven. Haar hulpvaardigheid, tips en vriendelijkheid hebben me geholpen om te blijven doorzetten. Ik mocht mij zeer gelukkig prijzen met haar als promotor.

Daarnaast dank ik mijn co-promotor Marjolijn De Jager. Zij heeft het mogelijk gemaakt om mijn onderzoek te kunnen verspreiden op de Haagse Hogeschool in Den Haag, Nederland. Ook wanneer er vaak weinig respons kwam van verschillende faculteiten, heeft ze mij geholpen om de juiste tussenpersonen te contacteren.

Vervolgens dank ik ook van de Hogeschool Gent Denis Amelynck, die me heeft geholpen om extra informatie te verzamelen over de studenten van de Hogeschool Gent. Deze informatie is zeer nuttig geweest voor een goeie inschatting te kunnen maken van de hedendaagse student aan de Hogeschool Gent. Door zijn correcte analyses en data is het mogelijk geweest om juiste linken te leggen tussen de steekproefdata en de data die HoGent al in zijn bezit had.

Mevrouw Anita Bernard van de Hogeschool Gent wil ik zeer hartelijk bedanken voor haar inzet voor de uitvoering van het praktijkgedeelte van dit onderzoek. Zij heeft als het ware een extreem belangrijke rol opgenomen in mijn bachelorproef, waarvoor ik haar niet genoeg kan bedanken. Mijn respect voor haar gedrevenheid en passie is alleen nog maar gestegen de afgelopen maanden.

Als laatste wil ik mijn vriendin, broer en ouders bedanken voor de morele steun. Vaak was het niet makkelijk om vanuit Den Haag te moeten communiceren met twee hogescholen en tegelijkertijd te werken als stagiair bij IT4Success in Den Haag. Allen hebben het door middel van videochat, berichten en positieve steun dragelijker gemaakt om dit alles tot een goed einde te brengen.