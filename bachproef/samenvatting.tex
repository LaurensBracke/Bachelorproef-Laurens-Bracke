%%=============================================================================
%% Samenvatting
%%=============================================================================

% TODO: De "abstract" of samenvatting is een kernachtige (~ 1 blz. voor een
% thesis) synthese van het document.
%
% Deze aspecten moeten zeker aan bod komen:
% - Context: waarom is dit werk belangrijk?
% - Nood: waarom moest dit onderzocht worden?
% - Taak: wat heb je precies gedaan?
% - Object: wat staat in dit document geschreven?
% - Resultaat: wat was het resultaat?
% - Conclusie: wat is/zijn de belangrijkste conclusie(s)?
% - Perspectief: blijven er nog vragen open die in de toekomst nog kunnen
%    onderzocht worden? Wat is een mogelijk vervolg voor jouw onderzoek?
%
% LET OP! Een samenvatting is GEEN voorwoord!

%%---------- Nederlandse samenvatting -----------------------------------------
%
% TODO: Als je je bachelorproef in het Engels schrijft, moet je eerst een
% Nederlandse samenvatting invoegen. Haal daarvoor onderstaande code uit
% commentaar.
% Wie zijn bachelorproef in het Nederlands schrijft, kan dit negeren, de inhoud
% wordt niet in het document ingevoegd.

\IfLanguageName{english}{%
\selectlanguage{dutch}
\chapter*{Samenvatting}
\lipsum[1-4]
\selectlanguage{english}
}{}

%%---------- Samenvatting -----------------------------------------------------
% De samenvatting in de hoofdtaal van het document

\chapter*{\IfLanguageName{dutch}{Samenvatting}{Abstract}}

De huidige generatie hogeschoolstudenten kan niet meer zonder zijn coole snufjes. Of het nu een smartphone, laptop of tablet is, elke student komt dagelijks wel met één of meer van deze apparaten in zijn rugzak naar school. Ook worden deze apparaten tegenwoordig omschreven als 'standaardgereedschap' om een volledige les te kunnen volgen. Maar is dit wel altijd nuttig? 

Of het nu leerkrachten zijn in België of in Nederland, ze klagen allemaal over de dalende interesse in hun lessen. Leerlingen hebben door dit volledig arsenaal aan devices een manier om voor een korte periode de les te 'verlaten' en samen met vrienden te lachen met een filmpje op sociale media, of een berichtje te sturen naar hun geliefde. Dit alles komt natuurlijk niet de kwaliteit van het lesgeven ten goede. Het is niet goed voor de leerkracht zijn moreel, en misschien ook niet voor de punten van de student... 

Door middel van een bevraging rond te sturen voor studenten van 4 verschillende studierichtingen op zowel de Haagse Hogeschool als de Hogeschool Gent, ben ik nagegaan wat de meningen zijn van de studenten over hun smartphonegebruik, of de tijd die ze spenderen achter een beeldscherm. Daarnaast is ook gevraagd naar de huidige resultaten die zij scoren in de examenperiode: hebben ze veel onvoldoendes, welk punt halen ze gemiddeld...? Aangezien studenten kunnen liegen in een bevraging die verdeeld wordt via sociale media en mail, is er ook een praktijktestje uitgevoerd bij studenten van de richting Toegepaste Informatica aan de Hogeschool Gent, waarbij studenten aan het einde van een theoretische les een aantal vragen over de net gegeven les moesten beantwoorden. In sommige klassen hadden studenten alle elektronische gadgets mogen gebruiken een hele les, in andere klassen werd opgelegd om enkel pen en papier een hele les te gebruiken.

Waar eerst in dit onderzoek een stand van zaken is gegeven over de verschillende studierichtingen aan beide hogescholen, wat men kan terugvinden in de vakliteratuur en welke trends er al zijn, is daarna een methodologie geschreven over hoe dit onderzoek in zijn werk is gegaan, gevolgd door de verschillend resultaten. Als laatste kan u een conclusie terugvinden met een kijk naar de toekomst.

Onderzoeksvragen die beantwoord werden door online enquête en praktijktest (DEEL 2):
\begin{itemize}
	\item Heeft het gebruik van smartphones en laptops tijdens de lessen mogelijks een invloed op de slaagcijfers en kennis van de student? (Hoofdvraag)
	\item Heeft geslacht, relatiestatus of leeftijd van de student een invloed op de aantrekkingskracht die deze devices hebben op de student?
	\item Is er een verschil in omgang met devices tussen Nederlandse en Vlaamse studenten?
	\item Wat denken studenten zelf over de invloed van smartphones op hun resultaten?
	\item Is er een verschil te merken tussen studenten uit een sociale richting en studenten uit een richting die IT-gerelateerd is?
	\item Is er een direct positief effect op de student in het klaslokaal wanneer de smartphone en laptop volledig uit beeld verdwijnen?
\end{itemize}

De resultaten van de online vragenlijst geven ons een indicatie voor de onderzochte populatie. Hieruit blijkt dat er vooral het gedrag van studenten tijdens de les een bepalende factor kan zijn voor de studieresultaten. Het aantal uur dat de student buiten de les spendeert op zijn smartphone of laptop is weinig of niet relevant. Hoe minder je je smartphone gebruikt, hoe beter je kans op slagen of een goed resultaat. Daarnaast konden we ook vaststellen dat vrouwen bijna een uur meer per dag spenderen op hun smartphone, dat vrijgezellen een half uur langer per dag hun smartphone of laptop gebruiken, dat Vlamingen meer op hun smartphone zitten maar dat bijna 9 op de 10 Nederlanders hun laptop meebrengen naar de les, dat 6 op 10 studenten aangeeft dat smartphones en laptops tijdens de les hen afleidt en dat studenten uit sociale richtingen 75 minuten per dag minder achter een scherm spenderen dan it-studenten.

De praktijktest bracht aan het licht dat een student uit een klas waar geen smartphones of laptops toegelaten zijn gemiddeld 5,6 procent meer kennis opnemen over de net gegeven les dan wanneer de student in een klas zat waar smartphone en laptop toegelaten waren, en dat, zonder te kijken naar in wat voor klas de student zit, een student die geen elektronica gebruikt de hele les 8,3 procent meer kennis heeft over de les dan een student die ervoor heeft gekozen om zijn smartphone of laptop te gebruiken.

Ondanks dat vele van deze resultaten aantonen dat studenten voor hun kennis en schoolresultaten beter deze apparaten zoveel mogelijk mijden wanneer ze binnen de schoolmuren komen, is er verder onderzoek nodig om echt significante resultaten te verkrijgen over de aangekaarte onderwerpen binnen deze bachelorproef. Er is vooral nood aan verder onderzoek over gedrag van studenten binnen de muren van de klas. Daar wordt kennis opgedaan die later kan beslissen over een positief of negatief resultaat voor de student zijn examens en zijn toekomst. 
