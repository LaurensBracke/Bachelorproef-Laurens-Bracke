%%=============================================================================
%% Inleiding
%%=============================================================================

\chapter{Inleiding}
\label{ch:inleiding}

\section{Probleemstelling}
\label{sec:probleemstelling}

BYOB (Bring your own device), m-learning, e-learning, mobile learning… Dit zijn maar enkele termen die vaak gebruikt worden op middelbare of hogere scholen. Al deze termen hebben 1 gemeenschappelijk doel: proberen om de student een optimale leerervaring te bieden, of deze zich nu op school voordoet of thuis. Deze apparaten, waar onmetelijk veel kennis op beschikbaar is, kunnen in de ogen van de schooldirectie mogelijks dienen als het beste studiekameraadje van de student. Deze nieuwe vriend kan namelijk voor de meeste problemen een pasklare oplossing bieden en af en toe ook de student een klein verzetje bieden (ontspanning, hobby’s…). Maar is dit wel zo? Als men de cijfers van de Haagse Hogeschool erbij neemt voor de opleiding HBO-ICT, ziet men dat maar 32 procent van alle studenten die starten aan de opleiding ook werkelijk aan hun diploma geraken binnen de 5 jaar (voor een opleiding van 4 jaar) \autocite{Studiekeuze2017}. Deze opleiding is een richting waar laptops, smartphones en andere elektronische apparaten veel sneller getolereerd worden dan bij andere opleidingen, al is het maar dat voor een ICT-richting deze apparaten een absolute vereiste zijn/worden om de lessen te kunnen volgen. Men kan zich dan de vraag stellen of deze apparaten wel degelijk gebruikt worden door de student als studiegereedschap. Hoe vaak heb je zelf als hedendaagse student eens rondgekeken in de klas en andere mensen betrapt dat ze bezig zijn met andere zaken zoals sociale media, berichten sturen naar hun beste vriend/vriendin? Misschien heb je wel jezelf al vaker betrapt op je onoplettendheid tijdens de les door randzaken? De zin “Weet jij op welke pagina we zitten?” klinkt voor de meeste studenten die deze tekst lezen waarschijnlijk niet onbekend in de oren. Zijn smartphones en laptops wel een motivatie om extra op te letten of te studeren voor een schoolvak? Vormen ze niet meer een afleiding voor de student? 

Dit onderzoek wil proberen aantonen dat het gebruik van smartphones en laptops tijdens de lessen een impact heeft op de studieresultaten of de kans op een onvoldoende voor een vak vergroot. Uit onderzoek bleek dat bijvoorbeeld jonge personen in Australië bijna allemaal een smartphone bezitten. Zo heeft 86 procent van alle mensen uit de leeftijdsgroep 18-24 jaar een smartphone op zak daar, terwijl in de groep 25-29 jaar zelfs 91 procent van de bevolking een smartphone bezit \autocite{Farley2015}. Als men dichter bij huis kijkt, blijkt uit het tweejaarlijks “Apestaartjaren-onderzoek” in België bij 3291 jongeren tussen de 12 en 18 jaar dat in 2016 wel 92 procent van deze leeftijdsgroep in het bezit is van een smartphone, waar dat in 2014 nog maar 86 procent was \autocite{Apestaartjaren2016}.

Andere bronnen \autocite{MatarBoumosleh2017} vermelden dat smartphones meer en meer een verslaving worden voor personen, die kunnen leiden tot depressies en angsten bij de studenten. Tegenwoordig zien de studenten de smartphone of laptop ook steeds vaker als bron van ontspanning in plaats van een werkinstrument \autocite{Baert2018}. \textcite{Baert2018} vermeldt ook het feit dat studenten zo vaak afgeleid zijn door het fenomeen FOMO (the fear of missing out). Studenten willen niets missen uit hun omgeving en moeten van alle updates op de hoogte zijn. De smartphone gebruiken voor sociale media blijkt meer en meer een eis te zijn om tot een vriendengroep of community te (blijven) behoren. Dit onderzoek wenst zich niet verder te focussen op wat de mogelijke oorzaken zijn van FOMO of verslaving of angsten door de smartphones en laptops, wel wilt het een aanzet geven tot verder onderzoek waarbij men kijkt of smartphones en laptops een aanleiding kunnen zijn tot bovenstaande termen, die vervolgens een effect hebben op de student zijn resultaten. Uit dit onderzoek moet blijken of het (overmatig) gebruik enkel en alleen tijdens de lessen ook een negatief effect met zich meebrengt voor de schoolresultaten of huidige kennis.

\section{Onderzoeksparameters}
\label{sec:onderzoeksparameters}

Als doelpubliek gebruikt men in dit onderzoek studenten die een hogere opleiding genieten, zowel in België als in Nederland. In dit onderzoek laten we de universitaire studies buiten beschouwing. De resultaten van dit onderzoek, hoe positief of negatief deze ook zijn, moeten de student wakker schudden en een zelfreflectie doen maken over hoe hij zijn of haar studiecarrière aanpakt. Hoewel het mogelijks kan lijken uit volgende hoofdstukken, heeft deze paper niet het doel om de scholen of de lesgevers aan te vallen. Zij proberen al eeuwenlang om alle mogelijke afleidingen voor studenten te bannen uit hun klaslokalen. Het zijn de studenten zelf die verantwoordelijk zijn voor de tools die ze meenemen richting de klas, het is in het hoger onderwijs niet de leerkracht zijn taak om studenten de hele tijd te berispen en aan te manen om hun mobieltje weg te stoppen. Het onderzoek heeft als extra doel om de student aan het woord te laten (door middel van een enquête), waarbij hij of zij zelf aangeeft wat kan en niet kan omtrent deze kwestie.

De Hogeschool Gent (België) en de Haagse Hogeschool (Nederland) zijn 2 onderwijsinstituten met studenten die tot ons steekproefkader behoren. Deze twee scholen hebben toegestemd dat enkele van hun studenten zullen onderworpen worden aan de testmethodes uit dit onderzoek, die verder worden gespecifieerd in het hoofdstuk “Methodologie”. De keuze voor studenten uit twee verschillende landen werd gemaakt om te onderzoeken of er verschillen zijn tussen twee buurlanden. Deze twee buurlanden spreken dezelfde taal en delen ongeveer dezelfde culturele waarden. Wel blijkt uit \textcite{VanGaalen2017} dat in Nederland het gebruik van smartphones en laptops meer wordt getolereerd dan in België. Zo blijkt uit een gesprek met twee studenten HBO-ICT dat vele van hun leerkrachten geen aandacht schenken aan wat studenten doen tijdens de les. Nederlandse leerkrachten vinden dat zij niet tegen jongvolwassenen moeten vertellen hoe belangrijk het is om op te letten tijdens de les. In België heerst meer het gevoel van gedeelde verantwoordelijkheid, waarbij men de bedoeling heeft om die persoon erop aan te spreken om zijn toekomstig gedrag proberen te doen wijzigen.

Het onderzoek naar de mogelijke correlatie tussen gebruik van elektronische devices tijdens de lessen en de slaagcijfers/kansen van studenten uit het hoger onderwijs vormt dan ook de belangrijkste onderzoeksvraag van deze paper. Met andere woorden, heeft het gebruik van smartphones of laptops een invloed op de examenresultaten, parate kennis of slaagpercentages van studenten van verschillende richtingen aan de hogescholen? Het is hiernaast ook relevant om enkele deelvragen te stellen aan hen. Het doel is om na te gaan of er verschillen in mate van gebruik tijdens de lessen worden opgemerkt tussen de twee hogescholen. Daarnaast zal dit onderzoek ook kijken naar welke soort devices de meeste aantrekkingskracht hebben op studenten die in de les zitten. Verder kan in dit onderzoek een onderscheid gemaakt worden tussen studenten uit een opleiding die IT-minded is (in dit onderzoek de opleidingen ICT of Bedrijfsmanagement) en opleidingen die meer neigen naar het menselijke, het sociale (in dit onderzoek de opleidingen Sociaal Werk of Pedagogie).

Als slot zal deze bachelorproef door praktijkonderzoek kijken of het verbannen van al deze apparaten uit de lessen een positief effect heeft op de aandacht van de student en zijn of haar kortetermijngeheugen. In samenwerking met leerkrachten uit de in hierboven vernoemde opleidingen zal het onderzoek proberen nagaan of de ‘verslaafde’ student uit de 21ste eeuw ook nog een uur zonder enige vorm van elektronische afleiding kan, en het verbannen van devices de student vooruithelpt in zijn parate kennis over de net onderrichte les.

\section{Onderzoeksvragen}
\label{sec:onderzoeksvragen}

\begin{itemize}
	\item Heeft het gebruik van smartphones en laptops tijdens de lessen mogelijks een invloed op de slaagcijfers en (parate) kennis van de student? (Hoofdvraag)
	\item Heeft geslacht, relatiestatus of leeftijd van de student een invloed op de aantrekkingskracht die deze devices hebben op de student?
	\item Is er een verschil in omgang met devices tussen Nederlandse en Vlaamse studenten?
	\item Wat denken studenten zelf over de invloed van smartphones op hun resultaten?
	\item Is er een verschil tussen studenten uit een sociale richting en studenten uit een richting die IT-gerelateerd is?
	\item Is er een direct positief effect op de student in het klaslokaal wanneer de smartphone en laptop volledig uit beeld verdwijnen?
\end{itemize}

\section{Onderzoeksdoelstelling}
\label{sec:onderzoeksdoelstelling}

De verwachtingen voor dit onderzoek zijn dat studenten zelf zullen aangeven dat zij hun apparaten heel vaak voor andere doeleinden gebruiken dan les of school, en dat zelfs meer dan de helft van alle studenten denkt van zichzelf dat hij/zij minder afgeleid is door deze devices dan de gemiddelde student. Nederland zal volgens de verwachtingen ook slechter scoren dan België, zowel op het vlak van smartphone- of laptopgebruik als het verschil in schoolresultaten tussen een ‘oplettende’ student en een ‘afgeleide’ student (men scoort slechter wanneer de mogelijkheid voor afleiding er is, wanneer we deze wegnemen zal de student een hogere affiniteit tonen met de les). Ook wordt verwacht dat de discipline van de Vlaamse studenten hoger zal zijn. Het onderzoek hoopt ook te concluderen dat de studenten die hoger dan de gemiddelde student scoren op vlak van afleiding van smartphones en laptops tijdens de lessen, een slaagpercentage hebben op de afgelegde examens lager dan 70 procent (m.a.w. van alle vakken die de student in zijn schooljaar/semester heeft opgenomen, heeft deze student maar voor 70 procent van die vakken het creditbewijs verkregen). Tenslotte verwacht men, met ondersteuning van het praktijkonderzoek bij de specifieke opleidingen, dat theorieklassen waar e-learning (leersituatie waarbij de leerkracht kiest om de student te onderwijzen met behulp van technologieën van het internet) en het gebruik van smartphone en laptop tijdelijk verboden wordt een positief effect zal teweegbrengen bij de student zijn aandacht en zijn opvattingen en kennis over de net gegeven les.

\section{Opzet van deze bachelorproef}
\label{sec:opzet-bachelorproef}

De rest van deze bachelorproef is als volgt opgebouwd:

In Hoofdstuk~\ref{ch:stand-van-zaken} wordt een overzicht gegeven van de stand van zaken binnen het onderzoeksdomein, op basis van een literatuurstudie. Ook kijken we wat we kunnen te weten komen over studenten van zowel de Haagse Hogeschool als de Hogeschool Gent.

In Hoofdstuk~\ref{ch:methodologie} wordt de methodologie toegelicht en worden de gebruikte onderzoekstechnieken besproken om een antwoord te kunnen formuleren op de onderzoeksvragen.

In Hoofdstuk~\ref{ch:resultaten_vragen} gaat het onderzoek dieper in op de resultaten van de enquête die verspreid is onder de studenten van de Hogeschool Gent en de Haagse Hogeschool. Conclusies over dit onderzoek worden ook direct gemaakt, met het doel om deze met de praktijk te toetsen bij het praktijkonderzoek.

In Hoofdstuk~\ref{ch:resultaten_praktijk} wordt een overzicht gegeven van de resultaten van de testen die zijn rondgegaan bij de verschillende opleidingen bij beide hogescholen. Deze worden op het eind gekoppeld aan de resultaten van de enquête. 

In Hoofdstuk~\ref{ch:conclusie}, tenslotte, wordt het verdict gegeven en een antwoord geformuleerd op de gestelde onderzoeksvragen. Daarbij wordt ook een aanzet gegeven voor toekomstig onderzoek binnen dit domein.

