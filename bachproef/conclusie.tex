%%=============================================================================
%% Conclusie
%%=============================================================================

\chapter{Conclusie}
\label{ch:conclusie}

%% TODO: Trek een duidelijke conclusie, in de vorm van een antwoord op de
%% onderzoeksvra(a)g(en). Wat was jouw bijdrage aan het onderzoeksdomein en
%% hoe biedt dit meerwaarde aan het vakgebied/doelgroep? Reflecteer kritisch
%% over het resultaat. Had je deze uitkomst verwacht? Zijn er zaken die nog
%% niet duidelijk zijn? Heeft het onderzoek geleid tot nieuwe vragen die
%% uitnodigen tot verder onderzoek?


In dit hoofdstuk worden de verschillende onderzoeksvragen overlopen, met hun resultaten en conclusie. Tenslotte wordt nog een eindconclusie gegeven en wordt er een aanbeveling gegeven voor toekomstige onderzoeken of thesissen in dit vakgebied.

Op de hoofdvraag of het gebruik van smartphones en laptops (tijdens de lessen) een invloed kan hebben op de examenresultaten, kennis en slaagkansen van de student, moeten we deels positief en deels negatief antwoorden. Ja, er is zeker en vast een invloed op de kennis en resultaten wanneer we de student observeren in de omgeving van de klas. Het aantal keer dat een student afgeleid is door zijn smartphone of laptop tijdens de les en dat hij hiermee actief bezig is, heeft op korte termijn een invloed op zijn parate kennis en op lange termijn een kleine invloed op het gemiddeld examencijfer dat een student behaalt. Een student die nooit op zijn smartphone kijkt in de klas zal op het einde van de rit zelfs gemiddeld 5 procent beter scoren dan een student die 1 à 2 keer checkt, en zelfs bijna 10 procent beter dan studenten die meer dan 6 keer per les hun nieuwste updates nakijken. Ook heeft het gebeuren in de klas een zeer kleine invloed van 2 à 3 procent op het percentage vakken dat een student moet herkansen.

Daarnaast brengt een student die vrijgezel is gemiddeld een halfuurtje per dag langer door voor een beeldscherm dan een student die in een relatie is. Waar men net zou denken dat studenten in een relatie veel tijd spenderen achter hun smartphone door de vele berichten die ze met elkaar sturen of de videogesprekken die ze voeren. Blijkbaar heeft dus een vrijgezel meer tijd alleen met zijn smartphone, dan een verliefd persoon die tijd doorbrengt bij zijn of haar geliefde. Vrouwen zijn ook volgens het onderzoek minder geïntrigeerd door al deze elektronica. Vrouwen spenderen dagelijks een 7-tal uur achter een device, waar mannen een uurtje extra erbij doen. De meeste mannelijke studenten spelen vaak een spelletje als ze thuis zijn of in de bus zitten op weg naar huis. Sowieso worden technologische snufjes ook het meest door mannen begeerd.

Waar we eerst ook dachten dat er een verband zou zijn in verband met de leeftijd van een student, blijkt dit niet echt het geval te zijn. Studenten die geboren zijn in de jaren 90 tonen allemaal gelijkaardige resultaten en het is niet zo dat jongere studenten meer aangetrokken worden door deze elektronische apparaten dan oudere studenten.

Het elektronicagebruik bleek bij Vlaamse studenten zeer hoog ten opzichte van Nederlandse studenten, een verschil tussen beide groepen van 15 procent. Nederlanders zijn dan weer zeer verknocht aan hun laptop, en bijna 9 op de 10 brengt deze ook dagelijks mee naar de les. Door deze admiratie voor de laptop bij onze noorderburen is hun dagelijks gebruik ook anderhalf uur langer dan de Vlamingen. 

Wel verrassend bij dit onderzoek was de vraag van studenten naar een nieuwe manier om leerstof aan te brengen, zoals podcasts of online lessen kunnen volgen. Ook lessen herbekijken is iets waar vele studenten positief op reageren. Wel moet je je bij al deze methoden afvragen natuurlijk wat dan nog de taak is van de leerkracht. Op deze manier wordt de leerkracht dan een veredelde vlogger, die nieuwe updates post op een soort van YouTube-kanaal. Wel verwacht was dat de meerderheid van de studenten aangeeft dat smartphones en laptops afleiden tijdens de lessen en dat ze vooral voor sociale media en berichten worden gebruikt tijdens de les. Wel moet men concluderen dat wanneer men stellingen aanbiedt met een oneven aantal mogelijkheden om de antwoorden, studenten die twijfelen snel gaan kiezen voor de middelste optie. Op deze manier dwing je de student niet om een uitgesproken mening te geven over de verschillende stellingen. Dit kan mogelijks beter worden uitgevoerd in een toekomstig onderzoek.

IT-kennis van ouders blijkt ook geen bepalende factor te zijn om te bepalen of een student veel tijd actief spendeert achter zijn apparaten. Wel blijkt bij in zowel België als in Nederland dat het soort studie dat je doet bepaalt of je veel of weinig gebruik maakt van je smartphone. In Nederland haalt bijna 47 procent van de studenten die een sociale richting volgen wel meer dan 5 keer zijn smartphone boven tijdens een lesuur, bij de richtingen die IT-gerelateerd zijn blijkt dit maar rond de 32 procent te zijn. In België is het verschil tussen beide categorieën zelfs bijna 30 procent. 6 op 10 Vlaamse studenten uit een sociale richting haalt meer dan 5 maal zijn gsm boven tijdens een les, wat bijna toch wijst op een verslaving of een hunkering naar sociale media of nieuwe berichtjes.

Voor onze laatste onderzoeksvraag hadden we onze praktijktest ingeschakeld, en de resultaten die daaruit kwamen, bleken te bevestigen wat vooraf werd gedacht. Er was een verschil van 5,6 procent tussen de ongecontroleerde klassen en de gecontroleerde klassen. Klassen waar actief het gebruik verboden wordt, geven betere resultaten. Vooral de fractie studenten die normaal gezien weinig zouden opgestoken hebben van de net gegeven les, hadden veel meer bijgeleerd en opgenomen dan normaal. De strenge maatregel zal dus zoals verwacht niet de topstudenten helpen, maar wel de studenten die het normaal moeilijk hebben om goeie resultaten te halen.

Zowel bij de bevraging als de praktijktest moeten we rekening houden met het feit dat alle verzamelde resultaten opgeteld niet benoemd kan worden als statistisch significant. Ze zijn hoogstens een goede indicatie over wat er zich afspeelt in de onderzochte studierichtingen aan beide hogescholen.

Dit volledig onderzoek heeft aangetoond dat het veelvuldig gebruik tijdens de lessen een negatief effect met zich kan meebrengen. Wat de student buiten de schoolmuren doet, blijkt volgens dit onderzoek van minder of totaal geen belang. Het is daar waar de leerstof wordt onderricht via mondelinge communicatie dat we studenten kunnen helpen om beter te scoren op een test of een examen. Zelfs als uit verder onderzoek blijkt dat het verbieden werkelijk een positief effect heeft, of als er nieuwe leertechnieken met smartphones ontstaan die gunstige resultaten beloven, moeten leerkrachten en scholen dit ook durven implementeren, al is het maar in het belang van de student. Als men de student kan helpen, moet men dit proberen. 

Het advies aan toekomstige onderzoekers is dan ook dat ze verder moeten kijken naar het gedrag van studenten wanneer deze zich binnen de schoolmuren zich bevindt, want dat is de belangrijkste plaats om kennis op te doen voor later. Wanneer nog meer testen worden afgenomen door toekomstige onderzoekers naar het gedrag dat studenten vertonen met hun devices, kunnen scholen en studenten misschien een consensus bereiken over smartphones en laptops tijdens en naast de lessen waar zowel studenten als leerkrachten zich in kunnen vinden, om op die manier samen te proberen streven naar een betere manier voor het doorgeven van kennis en wijsheid.