%%=============================================================================
%% Methodologie
%%=============================================================================

\chapter{Methodologie}
\label{ch:methodologie}

In dit hoofdstuk wordt geschetst hoe het onderzoek te werk is gegaan om alle nodige informatie te verzamelen om op die manier een gefundeerde eindconclusie neer te schrijven. Tevens moest uit alle informatie die verzameld werd een toegepast antwoord gegeven worden op de hoofdvraag en al zijn deelvragen.

\section{DEEL 1: Populatie en steekproef}
\label{sec:popsteek}

Als populatie is gekozen voor alle studenten die studeren aan een hogeschool in België of Nederland. Er is bewust gekozen voor bacheloropleidingen van 3 of 4 jaar. Universitaire richtingen zijn meer gericht op het theoretische, waar HBO5-opleidingen (België) of Mbo-opleidingen (equivalent van Nederland) meer op het praktische gericht zijn. Een Hbo- of Bacheloropleiding biedt een mix aan tussen theoretische en praktijkgerichte vakken aan de student. Op die manier kan ook later in het onderzoek een verschil gemaakt worden tussen gebruik van smartphones en laptops bij vooral theorievakken of eerder praktijkvakken. 

Als steekproef is per land gekozen voor enerzijds 2 opleidingen waar het dagdagelijks gebruik van smartphones en laptops niet weg te denken is en anderzijds 2 andere opleidingen die weinig of geen vereisten hebben in verband met veelvuldig gebruik van smartphones of laptops, zowel niet in het studieleven als in het werkveld later. Ook is getracht om de richtingen in Nederland te doen gelijken op de studies die in België worden ondervraagd.

Als Vlaamse hogeschool is gekozen voor de Hogeschool Gent (wat had u anders gedacht?). HoGent biedt academische of professionele bachelors aan die normaliter 3 jaar in beslag nemen. Als Nederlandse Hogeschool is gekozen voor de Haagse Hogeschool in Den Haag, Nederland. Deze hogeschool biedt Hbo-opleidingen aan die gebruikelijk 4 jaar duren.

De studenten uit de online bevraging volgden één van volgende opleidingen:
\begin{itemize}
\item Bachelor in de Toegepaste Informatica (Hogeschool Gent, België)
\item Bachelor in het Bedrijfsmanagement (Hogeschool Gent, België)
\item Bachelor in het Sociaal Werk (Hogeschool Gent, België)
\item Bachelor in de Orthopedagogie (Hogeschool Gent, België)

\item Bachelor HBO-ICT (Haagse Hogeschool, Nederland)
\item Bachelor Bedrijfskunde (Haagse Hogeschool, Nederland)
\item Bachelor Social Work (Haagse Hogeschool, Nederland)
\item Bachelor Pedagogiek (Haagse Hogeschool, Nederland)
\end{itemize}

Studenten uit bovenstaande richtingen zijn steeds gewaarborgd geweest van volledige anonimiteit tijdens het volledige onderzoek. 

\section{DEEL 2: Verspreiding Enquête}
\label{sec:enquete}

Dit onderzoek is opgesplitst in 2 kleinere onderzoeken. Ten eerste is gekozen om een enquête te verspreiden onder de studenten uit voorgaande studierichtingen. In het tweede onderzoek is gekozen voor een test die voorgelegd wordt aan de student tijdens de lessen. Hieronder wordt eerst de online bevraging van de studenten besproken.

De enquête die studenten uit de 8 verschillende richtingen hebben ingevuld konden ze terugvinden via onderstaande link (link is nu gesloten, vragenlijst kan u terugvinden in bijlage C):

\href{https://forms.office.com/Pages/ResponsePage.aspx?id=DjH3XBoJxUus1ybHIdTMzaq6IAIJ5OtInh7h0Z9qRNVUMTJRV1E1MzUzN0RSVkRPUExUUzEzNk1HVC4u}{Link naar vragenlijst}

Deze enquête telt 20 vragen (waar de laatste vraag verdeeld is in 10 kleinere vragen). De online bevraging is telkens voor ongeveer een maand blootgesteld aan de studenten. Om het onderzoek tot bij de student te brengen, is ervoor gekozen om de vragenlijst op 3 verschillende manieren te verspreiden: door een link te plaatsen in de Facebookgroep van een bepaalde opleiding, door een bericht te plaatsen op het studentenportaal met link naar de vragenlijst van de betreffende opleiding of door het persoonlijk sturen van een bericht met de link via de schoolmailadressen van de studenten.

Sowieso is hier in dit onderzoek gekozen voor alle links te verspreiden via het internet. Dit omdat blijkt uit voorgaande literatuurstudie dat zo goed als alle studenten wel over een smartphone, laptop of tablet beschikken. Dit heeft mogelijks wel tot gevolg dat er systematische steekproeffouten in het onderzoek sluipen, door een deel van de populatie een extra voordeel te geven op deze manier.

Door het aanbieden van verscheidene mogelijkheden mag wel gesteld worden dat het onderzoek hier een zo aselect mogelijke steekproef probeert op te stellen voor de studenten uit de populatie. Iedere student heeft zelfs als hij niet behoort tot een bepaalde Facebookgroep toch een manier om de vragenlijst in te vullen via de link uit de schoolmail. Deze manieren van verspreiden geeft elke student evenveel kans om zijn mening te delen.

De vragenlijst telt zowel vragen op het nominaal en ordinaal meetniveau (relatiestatus, geboortejaar, studierichting, verblijfplaats tijdens de week, IT-kennis van ouders, akkoord-/niet akkoord-vragen…) als vragen op het ratio-en intervalniveau (aantal uren in totaal achter laptop per dag, aantal uren enkel voor school bezig achter device, gemiddeld cijfer voor examenresultaten, percentage gebuisde vakken…). Het spreekt dan ook voor zichzelf dat hoe meer antwoorden er zijn op al deze vragen, hoe beter deze vragenlijst de verwachtingswaarde voor de hele populatie zal benaderen (Centrale Limietstelling).

Voor de hoofdvraag wordt bijvoorbeeld de cijfers van de gemiddelde score die een student haalt of het percentage resultaten waarvoor de student een onvoldoende haalde vergeleken met het aantal keer dat een student bezig is met andere zaken op zijn smartphone of laptop tijdens de les of het aantal uur dat een student dagelijks actief is op een smartphone of laptop (Lineaire regressie, correlatiecoëfficiënten).

Uit de stellingvragen op het einde van de bevraging hoopt dit onderzoek vooral de oprechte mening van de student te halen over de devices op school. Hier is niet de bedoeling om verbanden te zoeken, het heeft enkel als bedoeling te kijken hoe studenten zich voelen ten opzichte van een bepaalde stelling over hun manier van leven of hun materialen. Deze stellingvragen zijn hetgeen wat deze bevraging zo vernieuwend maakt ten opzichte van andere onderzoeken: waar andere onderzoekers zich meer focusten op de harde feiten, probeert dit onderzoek te kijken naar wat de student zelf denkt, beseft of voelt bij deze technologische vooruitgang.

De resultaten van deze enquête zijn via een Excel via Microsoft Forms beschikbaar gemaakt voor de onderzoeker, die vervolgens via softwareprogramma’s zoals Excel of R gezocht heeft naar verbanden, trends of harde feiten via wiskundige statistische begrippen (Pearson-correlatiecoëfficiënt, Lineaire regressie, kruistabellen…) tussen alle verzamelde data. De resultaten van deze online vragenlijst zijn terug te vinden in volgend hoofdstuk.

\section{DEEL 3: Test in de Praktijk}
\label{sec:praktijk}

In het laatste deel van het onderzoek is het de bedoeling om studenten van de richting Toegepaste Informatica in België via een korte theoretische toets te ondervragen over de les die ze net hebben gekregen.

Het onderzoek gaat als volgt in zijn werk:

\begin{itemize}
	\item De onderzoeker spreekt eerst met enkele leerkrachten uit de opleiding Toegepaste Informatica die vooral theoretische lessen geven. Indien zij op onderstaand onderzoek positief reageren en willen meedoen, geeft de onderzoeker hen het nodige materiaal om hiermee aan de slag te gaan.
	\item Het doel is om uiteindelijk 2 leerkrachten van de hogeschool Gent te overtuigen. Zo heeft men in totaal tussen de 4 en 8 verschillende klassen die onderstaande test onwetend zullen voltooien.
	\item Met test wordt hier bedoeld: een schriftelijke ondervraging aan de studenten die aanwezig zijn van 10 vragen die verband hebben met de net onderwezen materie.
	\item Elke leerkracht kiest voor het afgesproken theoretisch vak 2 van zijn klassen (ongeveer 20 personen per klas) die hij de test zal voorleggen. Men verkiest dat de leerkracht zijn 2 klassen op dezelfde dag dezelfde materie voorlegt en dat er zo weinig mogelijk tijd is tussen het lesgeven in klas 1 en het lesgeven in klas 2. Op die manier is mondelinge communicatie tussen de 2 klassen uitgesloten.
	\item In klas 1 zegt de leerkracht absoluut niets over de test en begint de leerkracht de theorieles te geven in één stuk door, ongeacht wat de studenten in de klas doen: luisteren naar de leerkracht, op hun smartphone of laptop zitten tokkelen, naar buiten kijken… Enkel praten met andere studenten is verboden en mag afgebroken worden door de leerkracht.
	\item 15 minuten voor het einde van de les start je met het ronddelen/verspreiden van de test met de 10 vragen. De leerkracht wacht tot het einde van de les (wanneer iedereen de test heeft ingediend) om de ware toedracht te verklappen aan de studenten. Leerkracht kan ook extra meegeven aan de studenten dat zij de resultaten niet kan inkijken, enkel de onderzoeker.
	\item In klas 2 doet men hetzelfde, met 1 groot verschil. Bij deze klas verbant men het gebruik van smartphones, laptops of tablets. Leerkracht geeft de studenten mee dat tijdens deze les geen enkel apparaat mag bovengehaald worden en dat wanneer wordt opgemerkt dat de student toch met 1 van zijn devices bezig is, wordt deze de eerste keer vriendelijk gevraagd om zijn toestel weg te steken. Indien dit nogmaals voorvalt bij dezelfde student, wordt deze student gevraagd het lokaal te verlaten. Enkel pen en papier is toegestaan voor de studenten. Hierna geeft de leerkracht weer les en verbiedt de leerkracht mondelinge communicatie tussen de leerlingen, net zoals in klas 1. Ook geeft de leerkracht hier dezelfde test met dezelfde vragen als klas 1. 
	\item Na deze 2 lessen krijgt de onderzoeker een Excel-document terug met daarin alle scores van klas 1 en klas 2, zonder enige naam van een student maar wel met een geslacht bij elke score. De leerkracht heeft dus totaal geen weet hoeveel welke leerling heeft gescoord.
\end{itemize}

Deze test heeft de bedoeling om te kijken of studenten die absoluut geen kans hebben om afgeleid te zijn door hun apparaten een betere parate kennis hebben van de net gegeven les. Als de gemiddelde score van klas 1 lager is dan de gemiddelde score van klas 2, is dit deelonderzoek geslaagd en kan gesteld worden dat een strenger reglement aangaande smartphones en laptops tijdens de les een positief effect kan hebben op de net opgedane kennis en aandacht van de in de klas aanwezige student. Dit geeft dan ook direct een antwoord op de gelijkaardige deelonderzoeksvraag.

Het unieke van dit deelonderzoek is dat men probeert om de studenten op geen enkele manier te vertellen dat ze meedoen aan dit onderzoek. Stel dat de leerkracht de studenten voor de les al op de hoogte stelt van een test of dit onderzoek, dan zal de student automatisch meer interesse in de les vertonen dan normaal. Het is de bedoeling dat alles in de klas zo naturel en normaal mogelijk blijft tot het beëindigen van de test door alle studenten. Op die manier is de aandacht van de student zo goed als ongewijzigd ten opzichte van eender welke andere les op de hogeschool. 

\section{DEEL 4: Analyseren resultaten}
\label{sec:analyse}

Nadat zowel vragenlijst als praktijkonderzoek is afgelopen, wordt alle data door de onderzoeker verzameld in de vorm van CSV- of Excel-bestanden. De uitgebreide analyse van deze resultaten kan men terugvinden in volgend hoofdstuk.

Onderzoeksvragen die worden beantwoord door online enquête (DEEL 2):
\begin{itemize}
	\item Heeft het gebruik van smartphones en laptops tijdens de lessen mogelijks een invloed op de slaagcijfers en kennis van de student? (Hoofdvraag)
	\item Heeft geslacht, relatiestatus of leeftijd van de student een invloed op de aantrekkingskracht die deze devices hebben op de student?
	\item Is er een verschil in omgang met devices tussen Nederlandse en Vlaamse studenten?
	\item Wat denken studenten zelf over de invloed van smartphones op hun resultaten?
	\item Is er een verschil te merken tussen studenten uit een sociale richting en studenten uit een richting die IT-gerelateerd is?
\end{itemize}

Onderzoeksvragen die worden beantwoord door praktijktest (DEEL 3):
\begin{itemize}
	\item Heeft het gebruik van smartphones en laptops tijdens de lessen mogelijks een invloed op de slaagcijfers en kennis van de student? (Hoofdvraag)
	\item Heeft geslacht van de student een invloed op de aantrekkingskracht die deze devices hebben op de student?
	\item Is er een direct positief effect op de student in het klaslokaal wanneer de smartphone en laptop volledig uit beeld verdwijnen?
\end{itemize}


